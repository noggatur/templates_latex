% Обозначения
\renewcommand{\AA}{\ensuremath{\mathring{\text{А}}}}
\newcommand{\aver}[1]{\left<#1\right>}				% Среднее выборочное
\newcommand{\bra}[1]{\ensuremath{\langle #1|}} 		% Бра-вектор
\newcommand{\bracket}[1]{\ensuremath{\langle #1|#1\rangle}}% Квадрат модуля вектора состояния
\newcommand{\cket}[1]{\ensuremath{#1\rangle}}  		% Кэт вектор, если умножается на бра-
\newcommand{\celsius}{^{\circ}\mathrm{C}}			% Градус цельсия
\newcommand{\g}{\ensuremath{^\circ}}				% Градус
\newcommand{\const}{\ensuremath{\mathrm{const}}}	% Постоянная
\newcommand{\Div}{\mathop{\mathrm{div}}\nolimits}	% Дивергенция
\newcommand{\e}{\mathop{\mathrm e}\nolimits}		% Экспонента
\renewcommand{\ge}{\geqslant}                 		% Больше или равно
\renewcommand{\le}{\leqslant}                   	% Меньше или равно
\newcommand{\grad}{\mathop{\mathrm{grad}}\nolimits} % Градиент
\renewcommand{\Im}{\mathop{\mathrm{Im}}\nolimits}	% Мнимая часть
\renewcommand{\Re}{\mathop{\mathrm{Re}}\nolimits}	% Действительная часть
\newcommand{\Ra}{\mathop{\mathrm{Ra}}\nolimits}		% Число Релея
\newcommand{\Gr}{\mathop{\mathrm{Gr}}\nolimits}		% Число Грассгофа
%\newcommand{\Pr}{\mathop{\mathrm{Pr}}\nolimits}	% Число Прандтля
\newcommand{\Oint}{\oint\limits}                	% Интеграл по контуру с правильными пределами интегрирования
\newcommand{\rot}{\mathop{\mathrm{rot}}\nolimits} 	% Ротор
\newcommand{\sign}{\mathop{\mathrm{sign}}\nolimits}	% Сигнум
\newcommand{\Sum}{\sum\limits}						% Cумма с правильными пределами интегрирования
\newcommand{\Int}{\int\limits}          			% Интеграл с правильными пределами интегрирования
\newcommand{\IInt}{\mathop{{\int\!\!\!\int}}\limits}% Двойной интеграл с правильными пределами интегрирования
\newcommand{\Tr}{\mathop{\mathrm{Tr}}\nolimits} 	% След матрицы
\ifbool{boldvec}{
	\newcommand{\veci}{\vec{i}}                			% i-орт
	\newcommand{\vecj}{\vec{j}}                			% j-орт
	\newcommand{\veck}{\vec{k}}                   		% k-орт
	\renewcommand{\vec}{\textbf}						% Жирный вектор вместо стрелочки
}{
	\newcommand{\veci}{\vec{\imath}}                	% i-орт
	\newcommand{\vecj}{\vec{\jmath}}                	% j-орт
	\newcommand{\veck}{\vec{k}}                   		% k-орт
}
\newcommand*{\eng}[1]{{\it #1}}						% Иностранные слова
\newcommand*{\abr}[1]{{\small #1}}					% Аббревиатуры
\newcommand*{\s}{\ensuremath{\quad\Rightarrow\quad}}% Следовательно
\newcommand*{\eq}{\ensuremath{\quad\Leftrightarrow\quad}}% Эквивалентно

% Множества чисел
%\newcommand{\C}{\ensuremath{\mathbb{C}}}	% Комплексные
\newcommand{\I}{\ensuremath{\mathbb{I}}}	% Иррациональные
\newcommand{\N}{\ensuremath{\mathbb{N}}}	% Натуральные
\newcommand{\Q}{\ensuremath{\mathbb{Q}}}	% Рациональные
\newcommand{\R}{\ensuremath{\mathbb{R}}}	% Вещественные
\newcommand{\Z}{\ensuremath{\mathbb{Z}}}	% Целые

% Физические величины
\DeclareMathOperator{\E}{\mathscr{E}}	% ЭДС


% Макросы
\newcommand*{\ind}[1]		% Нижний индекс русскими буквами
	{_{\text{\scriptsize #1}}}
\newcommand*{\vind}[1]		% Верхний индекс русскими буквами
	{^{\text{\scriptsize #1}}}
\newcommand*{\indfrac}[2]	% Дробь русскими буквами
	{\raisebox{2pt}{$\frac{\mbox{\small $#1$}}{\mbox{\small $#2$}}$}}
\newcommand*{\frc}[2]		% Большая косая дробь
	{\raisebox{2pt}{$#1$}\big/\raisebox{-3pt}{$#2$}}
\input{../_source/letterspacing}
\newcommand*{\tracking}[2]	% Изменение апрошей
	{\mbox{\letterspace to #1\naturalwidth{#2}}}
\newcommand*{\fig}[1]		% Ссылка на рисунок
	{(см.~рис.~\ref{fig:#1})}
\newcommand*{\ang}[2]		% Угол между двумя векторами
	{(\lefteqn{\,\widehat{\phantom{#1\,#2}}}#1,#2)}
\newcommand*{\partder}[2]	% Частная производная
	{\dfrac{\partial #1}{\partial #2}}
\newcommand*{\dpartder}[2]	% Вторая частная производная
	{\dfrac{\partial^2 #1}{\partial #2^2}}
\newcommand*{\hm}[1]		% Перенос знаков в формулах (по Львовскому)
	{#1\nobreak\discretionary{}{\hbox{$\mathsurround=0pt #1$}}{}}


% Изобретаем велосипед
\newcommand*{\dotvec}[1]		% Производная вектора по времени
	{\savebox{\hght}{$\vec{#1}$}\dot{\raisebox{0pt}[.8\ht\hght]{$\vec{#1}$}}}
\newcommand*{\ddotvec}[1]	% Вторая производная вектора по времени
	{\savebox{\hght}{$\vec{#1}$}\ddot{\raisebox{0pt}[.8\ht\hght]{$\vec{#1}$}}}
\newcommand*{\tenz}[1]		% Тензор
	{\boldsymbol{#1}_{ik}} %[1]{\lefteqn{\VEC{#1}}	% \overset{\rotatebox{180}{$\mathchar"017E$}}{\raisebox{-0.3pt}{\phantom{#1}}}}
\newcommand*{\hsd}			% Нижнаяя полусфера
	{\tikz[scale=-.16]{
		\draw (1,0) arc [start angle=0,end angle=180,radius=1];
		\draw (1,0) arc [start angle=0, end angle=180, x radius=1, y radius=0.3];
		\draw (1,0) arc [start angle=0, end angle=-180, x radius=1, y radius=0.3];
	}}
\newcommand*{\hsu}			% Верхняя полусфера
	{\tikz[scale=.16]{
		\draw (1,0) arc [start angle=0,end angle=180,radius=1];
		\draw[densely dashed] (1,0) arc [start angle=0, end angle=180, x radius=1, y radius=0.3];
		\draw (1,0) arc [start angle=0, end angle=-180, x radius=1, y radius=0.3];
	}}
\newcommand*{\sptr}			% Сферический треугольник
	{\raisebox{-.5pt}{\tikz[scale=1.2]{
		\draw (0,0) arc [start angle=0, end angle=-60,radius=0.23];
		\draw[xscale=-1] (0,0) arc [start angle=0, end angle=-60,radius=0.23];
		\draw (.114,-.2) arc [start angle=60, end angle=120,radius=0.23];
	}}}
\newcommand{\wtftop}{		% Верхняя часть непонятного куска
	{\hfill$\downarrow$ ЧТО С ЭТИМ ДЕЛАТЬ? $\downarrow$\hfill}
	\vspace{.1cm}\hrule\vspace{.7cm}}
\newcommand{\wtfbottom}{	% Нижняя часть непонятного куска
	\vspace{.7cm}\hrule\vspace{.2cm}
	{\hfill$\uparrow$ ЧТО С ЭТИМ ДЕЛАТЬ? $\uparrow$\hfill} }	% Пробел в конце обязателен!

\newcommand{\listeqn}{\hskip0pt\vspace{-1.2\baselineskip}}	% Сдиг уравнений, находящихся в списке (без поясняющего текста) вверх

% Макросы
\newcommand{\inputTwoFiguresHere}[8]{
	\begin{figure}[h!]
		\hfil
		\subfigure[#3]{#4\label{#5}}
		\hfil\hfil
		\subfigure[#6]{#7\label{#8}}
		\hfil
		\caption{#1}\label{#2}
	\end{figure}
}

% Окружения типа "Теорема"	
\theoremstyle{definition}
\newtheorem{example}{Пример}[chapter]