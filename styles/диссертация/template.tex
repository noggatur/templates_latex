% Установка языка документа
\usepackage{polyglossia}
\usepackage{ucharclasses}
\setdefaultlanguage{russian}
\setotherlanguages{english,greek}
\usepackage{fontspec}
\defaultfontfeatures{Mapping=tex-text}	% Дополнительные примочки ,Ligatures={TeX},Numbers=OldStyle
\newfontfamily{\cyrillicfont}{Tinos}
\setmainfont{Tinos}	% Нормальные: Charis SIL, Garamond Premier Pro (работают лигатуры), Gentium Plus (нет жирного, но очень хороший), Noto Serif
\newfontfamily{\cyrillicfontsf}{Open Sans}	% Шрифт без засечек
\setsansfont{Open Sans}	% Шрифт без засечек
\newfontfamily{\cyrillicfonttt}{Anonymous Pro}	% Моноширный шрифт
\setmonofont{Anonymous Pro}	% Моноширный шрифт
\setTransitionsForLatin{\begingroup\hyphenrules{english}}{\endgroup}	% Переносы английских слов
% Прямое начертание греческих символов

\renewcommand{\alpha}{\text{α}}
\renewcommand{\beta}{\text{β}}
\renewcommand{\gamma}{\text{γ}}
\renewcommand{\delta}{\text{δ}}
\renewcommand{\epsilon}{\varepsilon}
\renewcommand{\varepsilon}{\text{ε}}
\renewcommand{\zeta}{\text{ζ}}
\renewcommand{\eta}{\text{η}}
\renewcommand{\theta}{\text{θ}}
%\renewcommand{\vartheta}{\text{}}
\renewcommand{\iota}{\text{ι}}
\renewcommand{\kappa}{\text{κ}}
%\renewcommand{\varkappa}{\text{}}
\renewcommand{\lambda}{\text{λ}}
\renewcommand{\mu}{\text{μ}}
\renewcommand{\nu}{\text{ν}}
\renewcommand{\xi}{\text{ξ}}
\renewcommand{\pi}{\text{π}}
%\renewcommand{\Pi}{\text{Π}}
\renewcommand{\rho}{\text{ρ}}
\renewcommand{\sigma}{\text{σ}}
\renewcommand{\tau}{\text{τ}}
\renewcommand{\upsilon}{\text{υ}}
\renewcommand{\phi}{\varphi}
\renewcommand{\varphi}{\text{φ}}
\renewcommand{\chi}{\text{χ}}
\renewcommand{\psi}{\text{ψ}}
\renewcommand{\omega}{\text{ω}}
% Проверка (вставить в документ и раскомментировать)
% $
% \mathrm{A}\alpha~
% \mathrm{B}\beta~
% \Gamma\gamma~
% \Delta\delta~
% \mathrm{E}\epsilon\,\varepsilon~
% \mathrm{Z}\zeta~
% \mathrm{H}\eta~
% \Theta\theta\,\vartheta~
% \mathrm{I}\iota~
% \mathrm{K}\kappa~
% \Lambda\lambda~
% \mathrm{M}\mu~
% \mathrm{N}\nu
% $\\
% $
% \Xi\xi~
% \Pi\pi\,\varpi~
% \mathrm{P}\rho\,\varrho~
% \Sigma\sigma\,\varsigma~
% \mathrm{T}\tau~
% \Upsilon\upsilon~
% \Phi\phi\,\varphi~
% \mathrm{X}\chi~
% \Psi\psi~
% \Omega\omega
% $	% Прямое начертание греческих символов

% !!! Исправить левую весячую пунктуацию
\usepackage{microtype}	% Висячая пунктуация (работает только справа, почему-то)

\tolerance=400

%\usepackage{indentfirst}	% Задаёт отступ первому абзацу в новом разделе (в этом нет необходимости, потому что и так видно, что начался новый абзац)
\setlength{\parindent}{19pt}

\usepackage{setspace} % Интерлиньяж
%\onehalfspacing % Интерлиньяж 1.5
%\doublespacing % Интерлиньяж 2
%\singlespacing % Интерлиньяж 1

% Отступы от краёв страниц
\usepackage{geometry}
\geometry{left=2cm}
\geometry{right=1.5cm}
\geometry{top=1.5cm}
\geometry{bottom=2.2cm}

\usepackage{pfnote}		% Не сквозная нумерация сносок

% Настройка колонтитулов
\usepackage{fancyhdr}
\usepackage{nonumonpart}	% Убирает номер страницы с названием части
\pagestyle{fancy}
\lfoot[]{}\rfoot[]{}\rhead[]{}\lhead[]{}\chead[]{}
\cfoot[\small\thepage]{\small\thepage}
\renewcommand{\headrulewidth}{0pt}
\footskip=1cm

% Математика
\usepackage{amsfonts,amsmath,mathtools,amssymb,wasysym,marvosym,amsthm,ifsym}	% Шрифты, символы, окружения
\usepackage{euscript}	% Новый каллиграфический шрифт ($\mathcal{A} = \EuScript{A} \neq \CMcal{A}$)
\usepackage{mathrsfs}	% Шрифт с суперзавитушками для физики ($\mathscr{E}$)
%\usepackage{mathtext} 	% русские буквы в формулах
%\usepackage{t2}		% Возможность писать русский текст в формулах
%\usepackage{ionumbers}	% Разделителем в десятичных дробях становится точка
\usepackage{cases}		% Пояснения после уравнений системы (напр., "& x\in\Z")

% Списки
\usepackage{enumitem,iitem}
\setlist{topsep=5pt, leftmargin=*, itemsep=0pt, partopsep=0pt}
\setlist[1]{leftmargin=\parindent}
\setlist[2]{leftmargin=\parindent}
\makeatletter
	\AddEnumerateCounter{\asbuk}{\@asbuk}{м)}
\makeatother
	\renewcommand{\labelitemi}{--}
	\renewcommand{\labelenumi}{\arabic{enumi})}
	\renewcommand{\labelenumii}{\asbuk{enumii})}

% Цвета
\usepackage[usenames,dvipsnames]{xcolor}

% Рисунки и подписи
\usepackage{filecontents,graphicx,wrapfig,pgf,pgfplots,pgfplotstable,tikz,subfigure}
\graphicspath{{img/}{imgs/}}  % папки с картинками
%\setlength\fboxsep{3pt} % Отступ рамки \fbox{} от рисунка
%\setlength\fboxrule{1pt} % Толщина линий рамки \fbox{}
\usepackage[nooneline]{caption}
\captionsetup{margin=10pt,font=small,labelsep=period}	% Формат подрисуночной подписи\\
\captionsetup[table]{justification=raggedleft} 
\captionsetup[figure]{justification=centering}
\renewcommand{\thesubfigure}{\asbuk{subfigure})}	% Одна скобка при нумирации отдельных частей картинки
\pgfplotsset{compat=1.13}
\usepackage[european resistors,american currents]{circuitikz}	% Электрические схемы в TikZ
\usetikzlibrary{
	angles,
	arrows,
	arrows.meta,
	backgrounds,
	decorations.pathmorphing,
	decorations.pathreplacing,
	decorations.markings,
	calc,
	chains,
	fit,
	graphs,
	intersections,
	patterns,
	positioning,
	scopes,
	shapes,
	shapes.misc,
	shapes.symbols,
	through
}
\tikzstyle{axes}=[]
\tikzset{
	thick,>=stealth',
	font=\small,
	block/.style={rectangle,draw,fill=black!10,thick,align=center,minimum size=.5cm,text width=1cm},
	hl/.style={very thin,gray},
	dhl/.style={dashed,very thin,gray},
	gr/.style={fill=gray!20!white},
	every label/.style={black},
	fat dot/.style={insert path={node[circle, inner sep=1pt, fill=black]{}}},
	empty dot/.style={insert path={node[circle, inner sep=1pt, fill=white, draw=black]{}}},
	right angle to/.style 2 args={to path={(\tikztostart) -- (\tikztotarget) ($(\tikztotarget)!#2!(\tikztostart)$) -- ++($($(\tikztotarget)!#2!(#1)$)-(\tikztotarget)$) -- +($(\tikztotarget)-($(\tikztotarget)!#2!(\tikztostart)$)$) (\tikztotarget)}},% Дополнительный параметр указывает, в какую сторону повёрнут квадратик (нужно название узла на прямой); второй - размер квадратика
	right angle to/.default={0,0}{0pt},
	equal segment/.style={insert path={node[midway, sloped, yscale=.5] {$||$}}},
	equal segment1/.style={insert path={node[midway, sloped, yscale=.5] {$|$}}},
	equal segment3/.style={insert path={node[midway, sloped, yscale=.5] {$|||$}}},
	bisector/.style n args={3}{insert path={ % Бисектриса
			let
			\p1 = ($ (#2) - (#1) $),
			\p2 = ($ (#3) - (#1) $),
			\p3 = ($ (#3) - (#2) $),
			\n{a} = {.01*veclen(\x1,\y1)},
			\n{b} = {.01*veclen(\x2,\y2)},
			\n{c} = {.01*veclen(\x3,\y3)},
			% взята сотая часть длин векторов, чтобы в процессе просчёта длины биссектрисы не было больших значений
			% длина внутренней биссектрисы по теореме Стюарта + домножаем на 100
			\n{l} = {100*sqrt(\n{a}*\n{b}*(\n{a}+\n{b}+\n{c})*(\n{a}+\n{b}-\n{c}))/(\n{a}+\n{b})}
			in
			(#1) -- ($(#1)!\n{l}!($($(#1)!1cm!(#2)$)!.5!($(#1)!1cm!(#3)$)$)$)}},
	segment bisector/.style={to path={($(\tikztostart)!.5!(\tikztotarget)$) -- ($($(\tikztostart)!.5!(\tikztotarget)$)!#1!90:(\tikztotarget)$)}},	% Серединный перпендикуляр
	% Для числовых прямых
	p/.style={to path={(\tikztostart) -- (\tikztotarget) node[midway, above] {\tiny$+$}}},	% Плюс
	m/.style={to path={(\tikztostart) -- (\tikztotarget) node[midway, above] {\tiny$-$}}},	% Минус
	pd/.style={to path={(\tikztostart) -- (\tikztotarget) node[midway, below] {\tiny$+$}}},	% Плюс внизу
	md/.style={to path={(\tikztostart) -- (\tikztotarget) node[midway, below] {\tiny$-$}}},	% Минус внизу
	fd/.style={fat dot, insert path={node[below] {$#1$}}},	% Жирная точка
	ed/.style={empty dot, insert path={node[below] {$#1$}}},	% Пустая точка
	cu/.style={to path={(\tikztostart) rectangle ($(\tikztotarget)+(0,.08)$)}},	% Штриховка сверху
	cd/.style={to path={(\tikztostart) rectangle ($(\tikztotarget)+(0,-.08)$)}}, % Штриховка снизу
	point/.style={to path={(\tikztostart) .. controls +(90:.5) and +(90:.5) .. (\tikztotarget)}},	% Огибание от этой точки до другой точки
	to start/.style={insert path={.. controls +(90:.35) and +(0:1) .. +(-1,.35) -- ($(start)+(0,.35)$)}},	% Огибание от этой точки до начала
	to end/.style={insert path={.. controls +(90:.35) and +(180:1) .. +(1,.35) -- ($(end)+(0,.35)$)}}	% Огибание от этой точки до конца
}

% Цилиндрическая система координат
\makeatletter
\define@key{cylindricalkeys}{angle}{\def\myangle{#1}}
\define@key{cylindricalkeys}{radius}{\def\myradius{#1}}
\define@key{cylindricalkeys}{z}{\def\myz{#1}}
\tikzdeclarecoordinatesystem{cylindrical}%
{%
	\setkeys{cylindricalkeys}{#1}%
	\pgfpointadd
	{\pgfpointxyz
		{{\myradius*sin(\myangle)}}
		{\myz}
		{{\myradius*cos(\myangle)}}
	}
}

% Таблицы
\usepackage{makecell}
\usepackage{array,tabularx,tabulary,booktabs}	% Дополнительная работа с таблицами
\usepackage{longtable}	% Длинные таблицы
\usepackage{multirow}	% Слияние строк в таблице

% Рубрикация
\usepackage{titlesec}
\titleformat{\part}[display]
	{\Huge\flushright\thispagestyle{empty}}
	{\sf\MakeUppercase{Часть}~\thepart}
	{4pt}
	{\bf\MakeUppercase}
\titlespacing*{\chapter}{0pt}{0pt}{*4}
\titleformat{\chapter}[display]
	{\bf\Large\centering\thispagestyle{empty}}
	{\MakeUppercase{\chaptertitlename}~\thechapter}
	{0pt}
	{\titlerule[.8pt]%
	\vspace{1pt}%
	\titlerule%
	\vspace{1pt}\MakeUppercase}
\titlespacing*{\section}{0pt}{*4}{*1}
\titleformat{\section}
	{\bf\sf\small\raggedright\ifbool{sections_newpage}{\newpage}{}}
	{\thesection.}
	{4pt}
	{\uppercase}
\titlespacing*{\subsection}{\parindent}{*2}{*1}
\titleformat{\subsection}[block]
	{\bf\sf\small}
	{\thesubsection.}
	{4pt}
	{}
\titleformat{\subsubsection}[block]
	{\bf\sf\small}
	{}
	{4pt}
	{}
\renewcommand\cleardoublepage{}
%\renewcommand{\chaptermark}[1]{\markboth{#1}{#1}}	% Не знаю, зачем это

% Оглавление
\usepackage{shorttoc,titletoc}
\setcounter{tocdepth}{1}
\titlecontents{part}[1.5em]
	{\bf\large\vspace*{1.5em}}
	{}
	{\hspace*{1em}}
	{\titlerule*[4pt]{.}\contentspage}
\titlecontents{chapter}[1.5em]
	{\bf\small}
	{\contentslabel{1.5em}}
	{\hspace*{-1.5em}}
	{\titlerule*[4pt]{.}\contentspage}
\titlecontents{section}[4em]
	{\small}
	{\contentslabel{2.5em}}
	{\hspace*{-2.5em}}
	{\titlerule*[4pt]{.}\contentspage}
\titlecontents{subsection}[6em]
	{\small}
	{\contentslabel{2.5em}}
	{\hspace*{-2.5em}}
	{\titlerule*[4pt]{.}\contentspage}
\usepackage{tocbibind}	% Ссылки на библиографию, алфавитный указатель, списки таблиц и рисунков

% Библиография
%\usepackage{csquotes}
%\usepackage[square,numbers,sort&compress]{natbib}
%\renewcommand{\bibnumfmt}[1]{#1.\hfill} % нумерация источников в самом списке — через точку
%\setlength{\bibsep}{0pt}

\reversemarginpar		% Нормальные маргиналии
%\marginparsep=-20pt	% Расстояние между полем и маргиналиями

% Дополнительные элементы
%\usepackage{soul}				% Декорирование текста
\usepackage{epigraph}			% Эпиграф
\usepackage{lastpage}			% Последняя страница
\usepackage{ean13isbn}			% Штриховой код EAN13 с ISBN
\usepackage[marginpar]{todo}	% Список актуальных задач
\renewcommand{\todoname}{\large Проблемы и замечания:}
\renewcommand{\todomark}{\pddatt}
\newcommand{\pddatt}{\tikz[scale=1.2]{
	\draw[red,ultra thick,scale=.89,xshift=1pt,yshift=.65pt] (0,0) -- (.3,{.6*sin(60)}) -- (.6,0) -- cycle;
	\draw[rounded corners=.05cm,white,thick] (0,0) -- (.3,{.6*sin(60)}) -- (.6,0) -- cycle;
	\draw[rounded corners=.05cm] (0,0) -- (.3,{.6*sin(60)}) node [scale=1.3,below=3pt] {\textrm{\textbf{!}}} -- (.6,0) -- cycle;
}\\}

%	Сквозная нумерация
%\renewcommand{\thefigure}{\arabic{figure}}
%\renewcommand{\theequation}{\arabic{equation}}

% Компиляция
\usepackage{checkend}	% Более подробное сообщение об ошибке при наличии незакрытого окружения
%\mathtoolsset{showonlyrefs=true}	% Показывать ссылки напротив только тех уравнений, к которым есть ссылка (\eqref{}, просто с \ref{} не работает)